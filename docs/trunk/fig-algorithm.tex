% This is part of the Jabuti 1.0 Manual.
% Copyright 2003 (c) Auri Marcelo Rizzo Vicenzi, Marcio Eduardo Delamaro,
% Jose Carlos Maldonado.
% See the file FDL.TXT for copying conditions.

\begin{figure}[!ht]
{\cmdsize
\begin{center}
\begin{tabular}{|l|}\hline\\
\begin{minipage}{5in}
\noindent\# Input: $iG$, the instruction-CFG $GI = \langle NI, EI_p,EI_s, si, TI\rangle$ to be reduced; \\
\# Output: $bG$, the block CFG $bG = \langle N, E_p, E_s, s,
T\rangle$
\\ \ \\
%
01\bigind    s := NewNodeTo(si) \\
02\bigind    foreach $x \in N$ \\
03\bigind    \ind if $x$ has no successor \\
04\bigind    \ind\ind $T := T \cup \{x\}$ \\ \ \\
%
     \# Auxiliary function: NewNodeTo \\
     \# Input: A node $y$ from $iG$ \\
     \# Output: A node from $bG$ where $y$ has been inserted \\ \ \\
%
05\bigind     $ins$ := the instruction in $y$ \\
06\bigind     if the node $y$ has already been visited \\
07\bigind     \ind return the node $w \in N$ that contains $ins$ \\
08\bigind     $CurrentNode :=$ new block node \\
09\bigind     $N := N \cup \{CurrentNode\}$ \\
10\bigind     $x := y$ \\
11\bigind     repeat \\
12\bigind     \ind Include $x$ as part of $CurrentNode$ \\
13\bigind     \ind Mark $x$ as visited \\
14\bigind     \ind if $x$ should terminated the current block \\ % aqui � linha xx
15\bigind     \ind\ind foreach $v$ such that $(x,v) \in EI_p$ \\
16\bigind     \ind\ind\ind $E_p := E_p \cup \{(CurrentNode, NewNodeTo(v))\}$ \\
17\bigind     \ind\ind foreach $v$ such that $(x,v) \in EI_s$ \\
18\bigind     \ind\ind\ind $E_s := E_s \cup \{(CurrentNode, NewNodeTo(v))\}$ \\
19\bigind     \ind\ind $x := null$ \\
20\bigind     \ind else \\
21\bigind     \ind\ind if there exists a $v$ such that $(x,v) \in EI_p$ \\
22\bigind     \ind\ind\ind $x := v$ \\
23\bigind     \ind\ind else \\
24\bigind     \ind\ind\ind $x := null$ \\
25\bigind     while $x \neq null$ \\
26\bigind     return $CurrentNode$ \\
\end{minipage}\\\hline
\end{tabular}
\end{center}
}
\caption{Algorithm to generate the CFG from bytecode.}\label{fig:algorithm}
\end{figure}
