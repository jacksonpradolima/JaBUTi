% This is part of the Jabuti 1.0 Manual.
% Copyright 2003 (c) Auri Marcelo Rizzo Vicenzi, Marcio Eduardo Delamaro,
% Jose Carlos Maldonado.
% See the file FDL.TXT for copying conditions.

\begin{table}[!ht]
\caption{Different classes of bytecode
instruction.}\label{tab:classes}
\begin{center}\scriptsize
\begin{tabular}{|c|p{0.4\textwidth}|p{0.4\textwidth}|}\hline
\textbf{Class}&\textbf{Bytecode Instructions}&\textbf{Data-Flow Implication}\\\hline
0   & \setbci{\ctrl{athrow}, \ctrl{goto}, \ctrl{goto\_w},
    \ctrl{if\_acmpeq}, \ctrl{if\_acmpne}, \ctrl{if\_icmpeq}, \ctrl{if\_icmpge},
    \ctrl{if\_icmpgt}, \ctrl{if\_icmple}, \ctrl{if\_icmplt}, \ctrl{if\_icmpne},
    \ctrl{ifeq}, \ctrl{ifge}, \ctrl{ifgt}, \ctrl{ifle}, \ctrl{iflt}, \ctrl{ifne},
    \ctrl{ifnonnull}, \ctrl{ifnull}, \ctrl{lookupswitch}, \ctrl{tableswitch},
    \ctrl{areturn}, \ctrl{dreturn}, \ctrl{freturn}, \ctrl{ireturn}, \ctrl{lreturn}, \ctrl{return},
    \ctrl{ret},
    monitorenter, monitorexit, pop, pop2,
    breakpoint, impdep1, impdep2, nop,
    checkcast,
    wide, swap} &
    Instructions in this class have \textbf{no implication} for the flow of data.\\\hline
1   &  \setbci{\ctrl{invokeinterface}, \ctrl{invokespecial}, \ctrl{invokestatic}, \ctrl{invokevirtual},
    \ctrl{jsr}, \ctrl{jsr\_w},
    dadd, ddiv, dmul, dneg, drem, dsub, fadd, fdiv, fmul, fneg, frem, fsub,
    iadd, iand, idiv, imul, ineg, ior, irem, ishl, ishr, isub, iushr, ixor,
    ladd, land, ldiv, lmul, lneg, lor, lrem, lshl, lshr, lsub, lushr,
    lxor,
    arraylength, instanceof,
    aconst\_null, bipush, dconst, fconst, iconst, lconst, sipush,
    ldc, ldc\_w, ldc2\_w,
    d2f, d2i, d2l, f2d, f2i, f2l, i2b, i2c, i2d, i2f, i2l, i2s, l2d, l2f,
    l2i,
    new, multianewarray, anewarray, newarray,
    dcmpg, dcmpl, fcmpg, fcmpl, lcmp}
    &
    Instructions in this class have \textbf{no implication} for the flow of data.
    In addition, they leave an unknown element at the top of the operand stack.
    Accesses to such element will not characterize any use or definition. For example,
    instruction \bci{new} pushes an object reference onto the stack. Further use
    of such reference as a field access is not regarded as a definition or use.\\\hline
2   & \setbci{aaload, baload, caload, daload, faload,
    iaload, laload, saload}
    & Loads an array element to the top of the operand stacks and
    indicating the \textbf{use} of the array.\\\hline
3   & \setbci{aastore, bastore, castore, dastore,
    fastore, iastore, lastore, sastore}
    & Stores the value on the top of the operand stack
    into an array element, indicating the \textbf{definition} of the array.\\\hline
4   & \setbci{putfield}
    & Stores the value on the top of the operand stack into an
    instance field, indicating the \textbf{definition} of such an instance field.\\\hline
5   & \setbci{putstatic}
    & Stores the value on the top of the operand stack into a
    class field, indicating the \textbf{definition} of such a class field.\\\hline
6   & \setbci{dup, dup2, dup\_x1, dup\_x2, dup2\_x1, dup2\_x2}
    & Duplicates the value onto the top of the operand stack and
    has \textbf{no implication} for the flow of data.\\\hline
7   & \setbci{aload, dload, fload, iload, lload}
    & Loads the value of a given local variable onto the top
    of the operand stack, indicating a \textbf{use} of such a local variable.\\\hline
8   & \setbci{astore, dstore, fstore, istore, lstore}
    & Stores the value on
    the top of the operand stack into a local variable,
    indicating a \textbf{definition} of such a variable.\\\hline
9   & \setbci{getfield}
    & Loads an instance field onto the top of the operand stack,
    indicating a \textbf{use} of such an instance field.\\\hline
10  & \setbci{getstatic}
    & Loads a class field onto the top of the operand stack,
    indicating a \textbf{use} of such a class field.\\\hline
11  & \setbci{iinc}
    & Increments the value of a given local variable,
    indicating a \textbf{use} and a
    \textbf{definition} of such a local variable.\\\hline
\end{tabular}
\end{center}
\end{table}
